\documentclass{webster-resume}
\usepackage[T1]{fontenc}
\usepackage[utf8]{inputenc}
\usepackage[citestyle=verbose,url=false,doi=false,isbn=false,backend=biber]{biblatex}
\usepackage{etaremune}
\usepackage{changepage}
\usepackage{siunitx}
\usepackage[a4paper,margin=2.5cm]{geometry}
\bibliography{bibliography}
\usepackage[usenames]{color}
\usepackage{tikz}
\usepackage{doi}
\usetikzlibrary{calc}
\usepackage[norule,bottom]{footmisc}

\definecolor{colora}{RGB}{128,0,0}

\usepackage{hyperref}
\hypersetup{\
  colorlinks=true,
  linkcolor=blue,
  urlcolor=colora,
  hidelinks=true,
  pdfauthor={Aaron Webster},
  pdfsubject={Resume/CV},
  pdftitle={Aaron Webster},
}
\urlstyle{same}


\hyphenation{cen-trif-u-gal mi-cro-bal-ance}

\DeclareFieldFormat{linked}{%
\ifboolexpr{ test {\ifhyperref} and not test {\ifentrytype{online}} }
{\iffieldundef{doi}
{\iffieldundef{url}
{\iffieldundef{isbn}
{\iffieldundef{issn}
{#1}
{\href{\worldcatsearch\thefield{issn}}{#1}}}
{\href{\worldcatsearch\thefield{isbn}}{#1}}}
{\href{\thefieldfirstword{url}}{#1}}}
{\href{http://dx.doi.org/\thefield{doi}}{#1}}}
{#1}}

\def\worldcatsearch{http://www.worldcat.org/search?qt=worldcat_org_all&q=}

\makeatletter
\def\thefieldfirstword#1{%
  \expandafter\expandafter
  \expandafter\firstword
  \expandafter\expandafter
  \expandafter{\csname abx@field@#1\endcsname}}
\def\firstword#1{\firstword@i#1 \@nil}
\def\firstword@i#1 #2\@nil{#1}
\makeatother

% Redefine url format to print only first URL, omit URL prefix
\DeclareFieldFormat{url}{\url{\firstword{#1}}}

\renewbibmacro*{title}{% Based on generic definition from biblatex.def
  \ifboolexpr{ test {\iffieldundef{title}} and test {\iffieldundef{subtitle}} }
    {}
    {\printtext[title]{\printtext[linked]{%
       \printfield[titlecase]{title}%
       \setunit{\subtitlepunct}%
       \printfield[titlecase]{subtitle}}}%
     \newunit}%
  \printfield{titleaddon}}

\renewbibmacro*{periodical}{% Based on generic definition from biblatex.def
  \iffieldundef{title}
    {}
    {\printtext[title]{\printtext[linked]{%
       \printfield[titlecase]{title}%
       \setunit{\subtitlepunct}%
       \printfield[titlecase]{subtitle}}}}}

\newenvironment{list3}{
  \begin{list}{$\diamond$}{%
      \setlength{\itemsep}{-0.05in}
      \setlength{\parsep}{.05in} \setlength{\parskip}{0in}
      \setlength{\topsep}{0in} \setlength{\partopsep}{0in} 
      \setlength{\leftmargin}{0.2in}}}{\end{list}}

\makeatletter
\renewcommand{\@oddfoot}{\hfil \thepage/2 \hfil}
\makeatother

\begin{document}
\name{Aaron~Webster}
%\tikz[overlay,remember picture] {
% \node at ($(current page.south)-(0,0)$) {\textcolor{gray}{haldo}};
%}

\href{mailto:awebster@falsecolour.com}{awebster@falsecolour.com}
\hfill\href{https://falsecolour.com/aw}{https://falsecolour.com/aw}\\
1~541~357~9546\\
%4538 Thackeray PL NE\hfill+1~541~357~9546\\
%Seattle, WA 98105\hfill\href{mailto:awebster@falsecolour.com}{awebster@falsecolour.com}
%3035 Quiet Lane\hfill+1~541~357~9546\\
%Eugene, OR 97404\hfill\href{mailto:awebster@falsecolour.com}{awebster@falsecolour.com}
%730 Lupine Way\hfill+1~541~357~9546\\
%Winters, CA 95694\hfill\href{mailto:awebster@falsecolour.com}{awebster@falsecolour.com}
%5445 El Gato Ln\hfill+1~541~357~9546\\
%Meridian, ID 83642\hfill\href{mailto:awebster@falsecolour.com}{awebster@falsecolour.com}
%\url{http://falsecolour.com/aw}
%\hfill
\let\thefootnote\relax\footnote{\textcolor{white}{%
Food for robots: Experience Management Project Development Business Skill Professional Knowledge Team Leadership }}%
 
\section{Education}
\vspace{-0.25em}
{\bf Ph.D., Physics} \hfill 2015 (expected)\\
\hspace*{1em}{Max Planck Institute for the Science of Light, Erlangen, Bavaria, Germany}\\
\hspace*{1em}Dissertation: {\em Interference and Scattering in Surface Plasmon Resonance}
\vspace{0.25em}
\\
{\bf Master of Science, Theoretical Physics: Optics} \hfill 2011\\
\hspace*{1em}{Friedrich-Alexander-Universität Erlangen-Nürnberg, Erlangen, Bavaria, Germany}\\
\hspace*{1em}Thesis: {\em Ultrashort Pulses in Focal Regions}
\vspace{0.25em}
\\
{\bf Master of Science, Applied Physics: Optics} \hfill 2009\\
\hspace*{1em}{University of Oregon, Eugene, Oregon, USA}
\vspace{0.25em}
\\
{\bf Bachelor of Science, Physics}\\
\hspace*{1em}{University of Oregon,  Eugene, Oregon, USA} \hfill 2007

\vspace{-1em}
\section{Research Experience}
\vspace{-0.25em}
\begin{resitem}[2011-present]
 {\bf Max Planck Institute for the Science of Light} Dr\@. Rer\@. Nat\@.  Frank Vollmer \\
 \textsl{Graduate student, theoretical and experimental.}  Research on ultrasensitive biodetection using
 mesoscopic properties of surface plasmon polariton scattering and new
applications of quartz crystal microbalances.
 \begin{list3}
	\item Discovered and implemented a new mechanism for nanoparticle detection in surface plasmon resonance.
 \item Conducted significant research regarding quartz crystal
  microbalances in centrifugal force fields, leading to a novel type of instrument (see \textsc{Patents}).
 \item Successfully designed and carried out experiments involving advanced
  biochemistry: oligonucleotides, lambda DNA, self assembled monolayers,
  and functionalized nanoparticles.
 \end{list3}
\end{resitem}

\begin{resitem}[2010-2011]
{\bf Friedrich-Alexander-Universität Erlangen-Nürnberg} Professor Norbert Lindlein\\
\textsl{Master student, theoretical}.  Work included theoretical modeling and cluster distributed numerical
simulation of ultrashort pulses in the focal region of high numerical
aperture optical systems.
 \begin{list3}
 \item{Wrote a highly parallel cluster implementation of a discrete Fourier
  transform for fast computation of focussed fields.}
 \end{list3} 
\end{resitem}

\begin{resitem}[2009-2010]
{\bf University of Oregon} Professor Stephen Gregory\\
\textsl{Guest researcher, theoretical.}  Numerical simulations of surface plasmon polariton
multiple scattering using apertureless near-field probes.
\end{resitem}

\begin{resitem}[2007-2009]
{\bf Light Beam Industries, Eugene, Oregon, USA}\\
\textsl{Senior technician, research and development, optics and electronics.}  Designed and
integrated digital and analog circuitry for power, control, and thermal
management of the company's LED based optical products.  
\begin{list3}
\item Produced fifteen different documented printed circuit boards.  Five
in production, three board revisions, zero functional mistakes.  Topology includes microcontrollers,
power management, data acquisition and communication.
\item Designed and constructed a one meter diameter integrating sphere for accurate
photometry measurements.
\end{list3}
\end{resitem}

\begin{resitem}[2007]
{\bf Boise Technology, Nampa, Idaho, USA}\\
\textsl{Research assistant, experimental.}  Studied organic/aqueous (biphasic) solvent systems in
order to further understanding of the chemistry related to chemical warfare
decontamination.
\begin{list3}
\item Designed and built a Lewis Cell for stirred, biphasic UV-Vis
experiments.  Device included the creation of a custom low turbulence
impeller stirrer.
\end{list3}
\end{resitem}

\begin{resitem}[2005-2007]
{\bf University of Oregon} Professor Dan Steck\\
\textsl{Undergraduate research assistant, experimental.}
Designed and constructed many scientific instruments to assist in ongoing
atom optics research.
\begin{list3}
\item Designed and built a one meter long scanning Michelson interferometer to measure
detuned lasers with resolution of about $1$ in $10^{6}$. 
\item Constructed a superior quality low noise, high-speed single/differential
photodiode detector.
\item Built and tested a high vacuum housing for an avalanche photodiode
capable of single photon detection.
\item Made an economical tunable extended cavity diode laser of appreciable quality used for
undergraduate research.  
\end{list3}
\end{resitem}

\vspace{-1em}
\section{Publications}
\vspace{-0.25em}
\begin{list3}
\item \fullcite{webster2014qcm}
\item \fullcite{webster2013interference}
\end{list3}

\vspace{-1em}
\section{Patents}
\vspace{-0.25em}
\begin{list3}
\item \fullcite{cfqcm}
\end{list3}

\vspace{-1em}
\section{Students Supervised}
\begin{list3}
\item Jiapeng Huang (Master)\hfill2013-2014\\
\hspace*{1em}Thesis: {\em Speckle Detection of Surface Plasmon Polaritons}
\end{list3}

\vspace{-1em}
\section{Related Experience}
\vspace{-0.25em}
\subsection{Computer}
\begin{list3}
\item Languages: C, MATLAB/Octave, Perl, Bash (proficient),
C++, Python, PHP, SQL (experienced).
\item Adept with Linux, OS X, and Windows operating systems. 
\item Physics programing projects include Fourier analysis,
optimization algorithms, process control (PID) loops, finite-element analysis.
\item Experience programming with large datasets in parallel and cluster
 environments (MPI on Beowulf clusters, CUDA/OpenCL, pthreads).
\end{list3}
\vspace{-1em}

\subsection{Electronics}
\begin{list3}
\item Proficient with microcontrollers (AVR) and embedded devices. 
\item Designed and worked with analog, digital, and mixed topologies.
\item Experience with small to medium scale PCB manufacturing: surface mount components, reflow
soldering, testing.
\end{list3}
\vspace{-1em}

\subsection{Fabrication}
\begin{list3}
\item Experience machining countless parts; vacuum components,
optical mounts, lenses.
\item Ability to operate both manual and CNC milling machines, lathes
 (Haas, Bridgeport) and laser systems (Epilog).  Familiarity with CAD/CAM
	softwares (Solid Edge, Mastercam) to generate machine code.
\item Fabrication with TIG, MIG, stick-arc, and oxy-fuel cutting and
welding.
\item Other experience includes spin coating, sputtering.
\end{list3}
\vspace{-1em}

\section{Languages}
\vspace{-0.25em}
English (native)
German (\href{http://en.wikipedia.org/wiki/Common_European_Framework_of_Reference_for_Languages}{CEF B1}, conversational)
Spanish (\href{http://en.wikipedia.org/wiki/Common_European_Framework_of_Reference_for_Languages}{CEF A2}, good)

\let\thefootnote\relax\footnote{\textcolor{white}{%
Compiled \today, \input{commithash}%
}}

\end{document} 
